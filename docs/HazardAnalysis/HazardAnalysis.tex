\documentclass[12pt,letterpaper]{article}
\usepackage[utf8]{inputenc}
\usepackage[margin=1in]{geometry}
\usepackage[titletoc,title]{appendix}
\usepackage[normalem]{ulem}
\usepackage[shortlabels]{enumitem}
\usepackage{graphicx}
\usepackage{booktabs}
\usepackage{tabularx}
\usepackage{hyperref}
\usepackage{indentfirst}
\usepackage[dvipsnames]{xcolor}
\usepackage{fancyhdr}
\usepackage{pdflscape}
\usepackage{float}
\usepackage{multirow}
\usepackage{comment}
\usepackage{etoolbox}
\usepackage{enumitem}

\hypersetup{ colorlinks=true,       % false: boxed links; true: colored links
    linkcolor=red,          %
    citecolor=green,        % color of links to bibliography
    filecolor=magenta,      % color of file links
    urlcolor=cyan           % color of external links
}
\renewcommand{\thesubsection}{4.6}
\title{Hazard Analysis\\\progname}

\author{\authname}

\date{}

%% Comments

\usepackage{color}

\newif\ifcomments\commentsfalse %displays comments
%\newif\ifcomments\commentsfalse %so that comments do not display

\ifcomments
\newcommand{\authornote}[3]{\textcolor{#1}{[#3 ---#2]}}
\newcommand{\todo}[1]{\textcolor{red}{[TODO: #1]}}
\else
\newcommand{\authornote}[3]{}
\newcommand{\todo}[1]{}
\fi

\newcommand{\wss}[1]{\authornote{blue}{SS}{#1}} 
\newcommand{\plt}[1]{\authornote{magenta}{TPLT}{#1}} %For explanation of the template
\newcommand{\an}[1]{\authornote{cyan}{Author}{#1}}

%% Common Parts

\newcommand{\progname}{ProgName} % PUT YOUR PROGRAM NAME HERE
\newcommand{\authname}{Team \#, Team Name
\\ Student 1 name
\\ Student 2 name
\\ Student 3 name
\\ Student 4 name} % AUTHOR NAMES                  

\usepackage{hyperref}
    \hypersetup{colorlinks=true, linkcolor=blue, citecolor=blue, filecolor=blue,
                urlcolor=blue, unicode=false}
    \urlstyle{same}
                                


\begin{document}

\maketitle
\thispagestyle{empty}

~\newpage

\pagenumbering{roman}

\begin{table}[hp]
\caption{Revision History} \label{TblRevisionHistory}
\begin{tabularx}{\textwidth}{lll}
\toprule
\textbf{Date} & \textbf{Developer(s)} & \textbf{Change}\\
\midrule
Oct 13 2022 & Albert, Almen, David, Gary, Jonathan, Kabishan & Revision 0\\
\bottomrule
\end{tabularx}
\end{table}

~\newpage

\tableofcontents

~\newpage

\listoftables

~\newpage

\pagenumbering{arabic}

\section{Introduction}

This document is the hazard analysis of Park'd. A hazard is a property or
condition in the system together with a condition in the environment that has
the potential to cause harm or damage = loss (From Nancy Leveson's work).

\section{Scope and Purpose of Hazard Analysis}
The scope of the document is to identify the causes, effects, solutions, and new
requirements for possible hazards within the system boundaries.

\section{System Boundaries and Components}
The system boundary and components consists of the following:
\begin{itemize}
    \item Camera
    \begin{itemize}
        \item An external hardware that captures the video and images of parking
        lots and transmits this information as input for our machine-learning
        model
    \end{itemize}
    \item Park-d web application Band-end server system
    \begin{itemize}
        \item Communication System (Communication protocol library for different
        system components)
        \begin{itemize}
            \item A system responsible for the communication of the different
            components of the application.(Communicating through HTTP call or
            RPC call)
            \item A system responsible for recovery when a communication failure
            occurs.
    \end{itemize}
    \item Driver Navigation System
    \begin{itemize}
        \item A system Provides the user with the navigation information for our
        application when a user arrives at the parking lot
    \end{itemize}
    \item Administrative Map System
    \begin{itemize}
        \item A system specifically designed for parking lot owners to allow
        them to upload the physical layout of their parking lot
    \end{itemize}
    \item Machine-Learning Model
    \begin{itemize}
        \item This system serves to analyze the real-time video data of the
        parking lot and outputs vacant parking spot information upon user
        request.
    \end{itemize}
    \item Database Storage System
    \begin{itemize}
        \item The data storage system stores the necessary user information and
        parking lot information for our backend services
    \end{itemize}
    \end{itemize}
    \item Cloud server
    \begin{itemize}
        \item The cloud server provides the ability to host our services
        remotely, load balancing, and take web requests from users.
    \end{itemize}
    \item Local machine
    \begin{itemize}
        \item The local machine including a cellphone or laptop allows user to
        check parking lot information and sending requests to our services. 
    \end{itemize}
\end{itemize}

\section{Critical Assumptions}
There are no critical assumptions being made.

\newpage

\newgeometry{left=2cm,bottom=2cm}

\begin{landscape}
\begin{table}[hp]
\section{Failure Mode and Effect Analysis}

\begin{tabular}{|p{0.09\linewidth}|p{0.12\linewidth}|p{0.15\linewidth}|p{0.25\linewidth}|p{0.2\linewidth}|p{0.075\linewidth}|p{0.05\linewidth}|}
\hline
Functions & Failure Modes & Effects of Failure & Causes of Failure & Recommended
Actions & SR & Ref.\\
\hline
Navigation & No driving instructions provided & Driver cannot navigate to
desired parking space & Path finding algorithm ran out of memory to compute the
driving instructions & Inform the driver that driving instructions could not be
found and to try again later to avoid driver frustration &
\hyperref[isr13]{SR.13} & H1-1\\\cline{2-7} & Impossible driving instructions
provided & Driver cannot navigate to desired parking space & Path finding
algorithm provided directions that are blocked by obstacles & Allow driver to
report the obstacle and request another route &  \hyperref[isr4]{SR.4} &
H1-2\\\cline{2-7} & Lengthy driving instructions provided & Driver travels a
distance that exceeds the minimum distance required to reach the parking space &
Path finding algorithm could not find an optimal path in the requested amount of
time & Inform the driver that driving instructions with least travel could not
be found and to try again later to avoid driver frustration &
\hyperref[isr13]{SR.13} & H1-3\\
\hline
Spot detection & System classifies reserved or accessibility parking spaces as
normal parking spaces & Driver unknowingly parks in a parking space that is not
available to them & \begin{enumerate}[a., leftmargin=0.5cm, noitemsep,
nolistsep]\item Painted indicator for reserved or accessibility parking has
faded or obscured by nearby vehicles or shadows \item Some spaces are converted
to reserved spaces \end{enumerate} & Allow parking lot managers to edit the
parking lot layout to fix the errors &  \begin{enumerate}[a., leftmargin=0.5cm,
noitemsep, nolistsep]\item \hyperref[asr2]{SR.2}, \hyperref[asr3]{SR.3},
\hyperref[isr4]{SR.4} \item \hyperref[asr2]{SR.2}, \hyperref[asr3]{SR.3},
\hyperref[isr4]{SR.4}\end{enumerate}& H2-1\\\cline{2-7} & System unable to
detect any parking space & User cannot park at any parking space in the parking
lot & Weather conditions, such as snow, have hidden the parking space boundaries
& Use backup data if a large percentage of parking spots become obscured &
\hyperref[isr7]{SR.7} & H2-2\\
\hline
\end{tabular}
\caption{Failure Mode and Effect Analysis Table} \label{TblFMEA}
\end{table}

\begin{table}[hp]
\begin{tabular}{|p{0.09\linewidth}|p{0.12\linewidth}|p{0.15\linewidth}|p{0.25\linewidth}|p{0.2\linewidth}|p{0.075\linewidth}|p{0.05\linewidth}|}
\hline
Functions & Failure Modes & Effects of Failure & Causes of Failure & Recommended
Actions & SR & Ref.\\
\hline
Selecting Parking Space & System associates selection with wrong parking space &
\begin{enumerate}[a., leftmargin=0.5cm, noitemsep, nolistsep] \item Driver is
provided with wrong directions \item Wrong space is marked as occupied until
camera marks the space as still empty \end{enumerate} & Space database error;
Selection does not translate to the same space in the database & The system must
not deviate from the format it uses to store other parking spaces &
\hyperref[isr8]{SR.8} & H3-1\\\cline{2-7} & System allows selection of reserved
parking spaces while unauthorized & Driver is directed to spaces they are not
authorized to use & \begin{enumerate}[a., leftmargin=0.5cm, noitemsep,
nolistsep] \item Image recognition algorithm mislabels spot \item Interface
fails to hide unauthorized spaces \end{enumerate} & Invalid spaces should be
marked accordingly in the app; Spaces should be stored along with any of their
special properties & \begin{enumerate}[a., leftmargin=0.5cm, noitemsep,
nolistsep]\item \hyperref[isr9]{SR.9} \item \hyperref[isr9]{SR.9}
\end{enumerate}& H3-2\\
\hline
Parking Lot Mapping & System maps a parking space where there is none & Driver
is directed to park illegally & Image recognition algorithm fault & Allow for
manual corrections to the constructed map & \hyperref[asr2]{SR.2},
\hyperref[asr3]{SR.3} & H4-1\\\cline{2-7} & Recognized parking space is not
associated with the database & Gaps exist in the displayed map on a valid
parking space & Space database error; Parking space is recognized but not made
accessible in the database & Raise an error if the system fails to associate a
key with a given parking space & \hyperref[isr10]{SR.10} & H4-2\\\cline{2-7} &
Paths leading to parking spaces are not mapped & Driver cannot navigate to
desired parking space & \begin{enumerate}[a., leftmargin=0.5cm, noitemsep,
nolistsep] \item No driving instructions provided \item Driving instructions are
impossible to follow \end{enumerate} & Driving paths through the parking lot
should already be stored for any potential space & \begin{enumerate}[a.,
leftmargin=0.5cm, noitemsep, nolistsep]\item \hyperref[isr12]{SR.12} \item
\hyperref[isr12]{SR.12}\end{enumerate}& H4-3\\
\hline
\end{tabular}
\caption{Failure Mode and Effect Analysis Table, Part 2} \label{TblFMEA2}
\end{table}

\begin{table}[hp]
\begin{tabular}{|p{0.09\linewidth}|p{0.12\linewidth}|p{0.15\linewidth}|p{0.25\linewidth}|p{0.2\linewidth}|p{0.075\linewidth}|p{0.05\linewidth}|}
\hline
Functions & Failure Modes & Effects of Failure & Causes of Failure & Recommended
Actions & SR & Ref.\\
\hline
Viewing Parking Lot & System displays wrong parking lot layout & Driver does not
see accurate information about the layout of the parking lot & Fault in Parking
Lot Mapping algorithm & Allow driver to manually mark layout mistakes as
feedback to the system & \hyperref[isr4]{SR.4} & H5-1\\\cline{2-7} & System
displays wrong parking spot information & Driver does not see accurate
information about the status of parking spots & Fault in Parking Spot Detection
algorithm & Allow driver to manually mark parking spot status as feedback to the
system & \hyperref[isr4]{SR.4} & H5-2\\
\hline
Editing Parking Lot Layout & System does not save the changes made to the layout
& Parking lot manager is not able to apply changes they made to the parking lot
& Database not updated properly with changes made to the layout & Allow parking
lot manager to force an update to the layout stored in the database; Ensure
database is checking for manual changes to the layout of the parking lot &
\hyperref[isr5]{SR.5}, \hyperref[isr6]{SR.6} \hyperref[isr11]{SR.11} &
H6-1\\\cline{2-7} & System does not display tools to edit the layout of the
parking lot & Parking lot manager is not able to edit the layout of the parking
lot & System not displaying user interface for editing & Allow parking lot
manager to restart or refresh the editing view/interface &
\hyperref[isr5]{SR.5}, \hyperref[isr6]{SR.6}, \hyperref[isr11]{SR.11} & H6-2\\
\hline
\end{tabular}
\caption{Failure Mode and Effect Analysis Table, Part 3} \label{TblFMEA3}
\end{table}

\end{landscape}

\newpage

\restoregeometry

\section{Safety and Security Requirements}
The requirements that should be added to \progname's SRS based on the FMEA
analysis are written in \textcolor{red}{red}.
\subsubsection{Access Requirements}
\begin{enumerate}[{SR}1.] 
    \item The system's parking lot data shall be accessible only to the team and
    to the parking lot owner(s).\label{asr1} \\
    \textbf{Fit Criterion:} The data is password protected.
    \textcolor{red}{\item Only the parking lot owner(s) shall have the option to
    edit the parking space layout} \label{asr2} \\
    \textcolor{red}{\textbf{Fit Criterion:} The administrative console is the
    only view that has the option to edit the parking space}
    \textcolor{red}{\item Only the parking space manager(s) of a parking lot are
    allowed to have access to the administrative console for their parking lot}
    \label{asr3}\\
    \textcolor{red}{\textbf{Fit Criterion:} The administrative console of a
    parking lot can only edit and view analytics of the parking lot}
\end{enumerate}

\subsubsection{Integrity Requirements}
\begin{enumerate}[resume*] 
    \item The system shall prevent inaccurate data from being stored.
    \label{isr4}\\
    \textbf{Fit Criterion:} Stress test the system with accurate and inaccurate
    data and measure the data's accuracy. \textcolor{red}{\item Unsaved parking
    layout information should be stored locally if the information cannot be
    uploaded to the server} \label{isr5} \textcolor{red}{\item Unsaved parking
    layout information should attempt to upload to the server every 30 seconds}
    \label{isr6} \textcolor{red}{\item Parking layouts will be automatically
    backed up daily} \label{isr7} \textcolor{red}{\item No parking space should
    be stored in a different format in the database from other parking spaces}
    \label{isr8} \textcolor{red}{\item The system should only allow a parking
    spot to have 1 special property} \label{isr9}\\
    \textcolor{red}{\textbf{Fit Criterion:} A parking space is either labeled as
    accessibility, electric vehicle, or reserved} \textcolor{red}{\item Parking
    lot managers must be prompted when there is a failed attempt to add a
    parking spot to the database} \label{isr10} \textcolor{red}{\item Parking
    lot owners should be able to prompt the upload of their parking lot layout
    to the database and server} \label{isr11} \textcolor{red}{\item Correct
    paths should be stored to all parking spaces} \label{isr12}
    \textcolor{red}{\item Users are informed when an error occurs when the
    system is determining the navigation path} \label{isr13}
\end{enumerate}

\subsubsection{Privacy Requirements}
\begin{enumerate}[resume*] 
    \item The system shall ask for permission to use the driver's location
    data.\\
    \textbf{Fit Criterion:} The system has a driver location agreement form.
\end{enumerate}

\subsubsection{Audit Requirements}
\noindent \emph{N/A}

\subsubsection{Immunity Requirements}
\noindent \emph{N/A}

\section{Roadmap}
From the new safety and security requirements discovered, they are all
reasonable and can be implemented within the project time frame. SR1-13 are
expected to be implemented by Revision 1.

\end{document}