\documentclass[12pt, titlepage]{article}

\usepackage{amsmath, mathtools}

\usepackage[round]{natbib}
\usepackage{amsfonts}
\usepackage{amssymb}
\usepackage{graphicx}
\usepackage{colortbl}
\usepackage{xr}
\usepackage{hyperref}
\usepackage{longtable}
\usepackage{xfrac}
\usepackage{tabularx}
\usepackage{float}
\usepackage{siunitx}
\usepackage{booktabs}
\usepackage{multirow}
\usepackage[section]{placeins}
\usepackage{caption}
\usepackage{fullpage}
\usepackage{makecell}

\hypersetup{ bookmarks=true,     % show bookmarks bar?
colorlinks=true,       % false: boxed links; true: colored links
linkcolor=red,          % color of internal links (change box color with
%linkbordercolor)
citecolor=blue,      % color of links to bibliography
filecolor=magenta,  % color of file links
urlcolor=cyan          % color of external links
}

\usepackage{array}

\externaldocument{../../SRS/SRS}

%% Comments

\usepackage{color}

\newif\ifcomments\commentsfalse %displays comments
%\newif\ifcomments\commentsfalse %so that comments do not display

\ifcomments
\newcommand{\authornote}[3]{\textcolor{#1}{[#3 ---#2]}}
\newcommand{\todo}[1]{\textcolor{red}{[TODO: #1]}}
\else
\newcommand{\authornote}[3]{}
\newcommand{\todo}[1]{}
\fi

\newcommand{\wss}[1]{\authornote{blue}{SS}{#1}} 
\newcommand{\plt}[1]{\authornote{magenta}{TPLT}{#1}} %For explanation of the template
\newcommand{\an}[1]{\authornote{cyan}{Author}{#1}}

%% Common Parts

\newcommand{\progname}{ProgName} % PUT YOUR PROGRAM NAME HERE
\newcommand{\authname}{Team \#, Team Name
\\ Student 1 name
\\ Student 2 name
\\ Student 3 name
\\ Student 4 name} % AUTHOR NAMES                  

\usepackage{hyperref}
    \hypersetup{colorlinks=true, linkcolor=blue, citecolor=blue, filecolor=blue,
                urlcolor=blue, unicode=false}
    \urlstyle{same}
                                


\begin{document}

\title{Module Interface Specification for \progname{}}

\author{\authname}

\date{\today}

\maketitle

\pagenumbering{roman}

\section{Revision History}

\begin{tabularx}{\textwidth}{p{3cm}p{2cm}X} \toprule {\bf Date} & {\bf Version}
& {\bf Notes}\\
\midrule
Jan 18, 2023 & 1.0 & Revision 0\\
\bottomrule
\end{tabularx}

~\newpage

\section{Symbols, Abbreviations and Acronyms}

See SRS Documentation at
\href{https://github.com/parkd-app/park-d/blob/main/docs/SRS/SRS.pdf}{Park'd
Software Requirements Specification}.\\

\noindent
\renewcommand{\arraystretch}{1.2}
\begin{tabular}{l l} 
  \toprule		
  \textbf{Symbol} & \textbf{Description}\\
  \midrule 
  \progname & Parking Lot Application\\
  MIS & Module Interface Specification \\
  MG & Module Guide\\
  SRS & Software Requirements Specifications\\
  \bottomrule
\end{tabular}\\

\newpage

\tableofcontents

\newpage

\pagenumbering{arabic}

\section{Introduction}

The following document details the Module Interface Specifications for Park'd,
our parking assistant application. Park'd aims to help drivers find parking
spaces by using machine learning algorithms to locate empty spaces from overhead
cameras. Our application then directs drivers to those spaces, taking into
account restrictions like reserved or accessible spaces. It will maintain a
database of spaces as well as a navigation layout for a given parking lot.

Complementary documents include the SRS and MG documents. The full documentation
and implementation can be found at
\href{https://github.com/parkd-app/park-d/tree/main/docs/Design/MIS/MIS.pdf}{Module
Interface Specification} and
\href{https://github.com/parkd-app/park-d/tree/main/docs/Design/MG/MG.pdf}{Module
Guide} documents, respectively.

\section{Notation}

The structure of the MIS for modules comes from \emph{Software Design, Automated
Testing, and Maintenance: A Practical Approach} \cite{HoffmanAndStrooper1995},
with the addition that template modules have been adapted from
\emph{Fundamentals of Software Engineering} \cite{GhezziEtAl2003}.  The
mathematical notation comes from Chapter 3 of \emph{Software Design, Automated
Testing, and Maintenance: A Practical Approach} \cite{HoffmanAndStrooper1995}.
For instance, the symbol := is used for a multiple assignment statement and
conditional rules follow the form $(c_1 \Rightarrow r_1 | c_2 \Rightarrow r_2 |
... | c_n \Rightarrow r_n )$.

\newpage
The following table summarizes the primitive data types used by \progname. 

\begin{center}
\renewcommand{\arraystretch}{1.2}
\noindent 
\begin{tabular}{l l p{7.5cm}} 
\toprule 
\textbf{Data Type} & \textbf{Notation} & \textbf{Description}\\ 
\midrule
character & char & a single symbol or digit\\
integer & $\mathbb{Z}$ & a number without a fractional component in (-$\infty$,
$\infty$) \\
natural number & $\mathbb{N}$ & a number without a fractional component in [1,
$\infty$) \\
real & $\mathbb{R}$ & any number in (-$\infty$, $\infty$)\\
null & $\epsilon$ & empty value \\
Boolean & $\mathbb{B}$ & true or false\\
String & String & a sequence of characters\\
Seq & Seq & an ordered collection of elements\\
exists & $\exists$ & true if there exists an element that satisfies a property,
false otherwise\\
for all & $\forall$ & true if all elements satisfy a property, false otherwise\\
implies & $\Rightarrow$ & true if the left operator is true then output the
right operator, false otherwise\\
in & $\in$ & true if a an element is in a Seq\\
and & $\land$ & true if both operators are true, false otherwise\\
subset & $\subseteq$ & true if a set contains another, false otherwise\\
\bottomrule
\end{tabular} 
\end{center}

\noindent
The specification of \progname \ uses some derived data types: sequences,
strings, and tuples. Sequences are lists filled with elements of the same data
type. Strings are sequences of characters. Tuples contain a list of values,
potentially of different types. In addition, \progname \ uses functions, which
are defined by the data types of their inputs and outputs. Local functions are
described by giving their type signature followed by their specification.

\newpage
\section{Module Decomposition}

The following table is taken directly from the
\href{https://github.com/parkd-app/park-d/blob/main/docs/Design/MG/MG.pdf}{Module
Guide} document for this project.

\begin{table}[h!]
\centering
\begin{tabular}{p{0.3\textwidth} p{0.6\textwidth}}
\toprule
\textbf{Level 1} & \textbf{Level 2}\\
\midrule

{Hardware-Hiding Module} & Camera capture module \\
\midrule

\multirow{7}{0.3\textwidth}{Behaviour-Hiding Module} & Admin console module\\
& Admin module\\
& Parking lot layout module\\
& Parking layout element module\\
& Parking spot module\\
& Authentication module\\
& User module\\
& User action handler module\\
& Vehicle module\\
& View module\\
\midrule

\multirow{3}{0.3\textwidth}{Software Decision Module} & Navigation module\\
& Parking Stats module\\
& Machine learning model module\\
\bottomrule

\end{tabular}
\caption{Module Hierarchy}
\label{TblMH}
\end{table}

\newpage

\section{MIS of Authentication Module} 
\label{auth:Module}

\subsection{Module}
AuthT

\subsection{Uses}
\hyperref[user:Module]{UserT}

\subsection{Syntax}

\subsubsection{Exported Constants}
None

\subsubsection{Exported Types}
AuthT = ?

\subsubsection{Exported Access Programs}

\begin{tabular}{p{4cm} p{4cm} p{2cm} p{4cm}}
\hline
\textbf{Name} & \textbf{In} & \textbf{Out} & \textbf{Exceptions} \\
\hline
authenticateUser & String, String & UserT & MissingUserException \\
\hline
\end{tabular}

\subsection{Semantics}

\subsubsection{State Variables}
None

\subsubsection{Environment Variables}
\textit{users}: Seq of UserT 

\subsubsection{Assumptions}
None

\subsubsection{Access Routine Semantics}

\noindent authenticateUser(\textit{id}, \textit{pass}):
\begin{itemize}
\item output: $out := \exists (i : \mathbb{N} | i < |users| \land
users[i].getUserId() = \\ id \land users[i].getPassword() = pass)) \Rightarrow
users[i]$

\item exception: $exc := \neg \exists (i : \mathbb{N} | i < |users| \land
users[i].getUserId() = \\ id \land users[i].getPassword() = pass)) \Rightarrow
MissingUserException$
\end{itemize}

\newpage

\section{MIS of User Module} 
\label{user:Module}

\subsection{Template Module}
UserT

\subsection{Uses}
\hyperref[vehicle:Module]{VehicleT}

\subsection{Syntax}

\subsubsection{Exported Constants}
None

\subsubsection{Exported Types}
UserT = ?

\subsubsection{Exported Access Programs}


\begin{tabular}{p{2cm} p{5cm} p{2cm} p{4cm}}
\hline
\textbf{Name} & \textbf{In} & \textbf{Out} & \textbf{Exceptions} \\
\hline
new UserT & String, String, VehicleT & UserT & UserCreationException \\
getUserId & ~ & String & ~ \\
getPassword & ~ & String & ~ \\
getVehicle & ~ & VehicleT & ~ \\
\hline
\end{tabular}

\subsection{Semantics}

\subsubsection{State Variables}
\textit{userId}: String \\
\textit{password}: String \\
\textit{vehicle}: VehicleT

\subsubsection{Environment Variables}
None

\subsubsection{Assumptions}
None

\subsubsection{Access Routine Semantics}

\noindent new UserT(\textit{id}, \textit{pass}, \textit{veh}):
\begin{itemize}
\item transition: \textit{userId}, \textit{password}, \textit{vehicle} :=
\textit{id}, \textit{pass}, \textit{veh} 
\item output: \textit{out} := self
\item exception: exc := $((|id| = 0) \vee (|pass| = 0) \vee (veh = \epsilon)
\Rightarrow \text{UserCreationException})$
\end{itemize}

\noindent getUserId():
\begin{itemize} 
\item output: \textit{out} := \textit{userId} 
\item exception: none
\end{itemize}

\noindent getPassword():
\begin{itemize} 
\item output: \textit{out} := \textit{password} 
\item exception: none
\end{itemize}

\noindent getVehicle():
\begin{itemize} 
\item output: \textit{out} := \textit{vehicle} 
\item exception: none
\end{itemize}

\newpage

\section{MIS of Vehicle Module} 
\label{vehicle:Module}

\subsection{Template Module}
VehicleT

\subsection{Uses}
None

\subsection{Syntax}

\subsubsection{Exported Constants}
None

\subsubsection{Exported Types}
VehicleT = ?

\subsubsection{Exported Access Programs}

\begin{tabular}{p{4cm} p{3cm} p{2cm} p{5cm}}
\hline
\textbf{Name} & \textbf{In} & \textbf{Out} & \textbf{Exceptions} \\
\hline
new VehicleT & $\mathbb{R}$, $\mathbb{R}$ & ~ & VehicleCreationException \\
getLength & ~ & $\mathbb{R}$ & ~ \\
getWidth & ~ & $\mathbb{R}$ & ~ \\
\hline
\end{tabular}

\subsection{Semantics}

\subsubsection{State Variables}
\textit{length}: $\mathbb{R}$ \\
\textit{width}: $\mathbb{R}$

\subsubsection{Environment Variables}
None

\subsubsection{Assumptions}
None

\subsubsection{Access Routine Semantics}

\noindent new VehicleT(\textit{len, wid}):
\begin{itemize}
\item transition: \textit{length}, \textit{width} := \textit{len}, \textit{wid} 
\item output: \textit{out} := self
\item exception: exc := $((len \leq 0) \vee (wid \leq 0) \Rightarrow
\text{VehicleCreationException})$
\end{itemize}

\noindent getLength():
\begin{itemize} 
\item output: \textit{out} := \textit{length} 
\item exception: none
\end{itemize}

\noindent getWidth():
\begin{itemize} 
\item output: \textit{out} := \textit{width} 
\item exception: none
\end{itemize}

\newpage

\section{MIS of Admin Console Module} 
\label{adminConsole:Module}

\subsection{Module}
AdminConsoleT

\subsection{Uses}
\hyperref[admin:Module]{AdminT}

\subsection{Syntax}

\subsubsection{Exported Constants}
None

\subsubsection{Exported Types}
AdminConsoleT = ?

\subsubsection{Exported Access Programs}

\begin{tabular}{l l l l}
\hline
\textbf{Name} & \textbf{In} & \textbf{Out} & \textbf{Exceptions} \\
\hline
authenticateAdmin & String, String & AdminT & MissingAdminException \\
\hline
\end{tabular}

\subsection{Semantics}

\subsubsection{State Variables}
None

\subsubsection{Environment Variables}
\textit{admins}: Seq of AdminT 

\subsubsection{State Invariant}
None

\subsubsection{Assumptions}
None

\subsubsection{Access Routine Semantics}

\noindent authenticateAdmin($id, pass$):
\begin{itemize}
\item output: $out := \exists (i : \mathbb{N} | i < |admins| \land
admins[i].getAdminId() = id \land \\ admins[i].getPassword() = pass))
\Rightarrow admins[i]$
\item exception: $exc := \neg \exists (i : \mathbb{N} | i < |admins| \land
admins[i].getAdminId() = id \land \\ admins[i].getPassword() = pass))
\Rightarrow MissingAdminException$
\end{itemize}

\newpage

\section{MIS of Admin Module} 
\label{admin:Module}

\subsection{Template Module}
AdminT

\subsection{Uses}
\hyperref[parkingLotLayout:Module]{ParkingLotLayoutT},
\hyperref[parkingLayoutElem:Module]{ParkingLayoutElemT}, 
\hyperref[parkingSpot:Module]{ParkingSpotT}

\subsection{Syntax}

\subsubsection{Exported Constants}
None

\subsubsection{Exported Types}
AdminT = ?

\subsubsection{Exported Access Programs}

\begin{tabular}{l l l l}
\hline
\textbf{Name} & \textbf{In} & \textbf{Out} & \textbf{Exceptions} \\
\hline
new AdminT & String, String & AdminT & AdminCreationException \\
getAdminId & ~ & String & ~ \\
getPassword & ~ & String & ~ \\
getLayout & String & ParkingLotLayoutT & LayoutNotFoundException \\
changeLayout & String, ParkingLayoutElemT & ~ & LayoutNotFoundException \\
\hline
\end{tabular}

\subsection{Semantics}

\subsubsection{State Variables}
$adminId$: String \\
$password$: String \\
$layouts$: seq of ParkingLotLayoutT

\subsubsection{Environment Variables}
None

\subsubsection{State Invariant}
None

\subsubsection{Assumptions}
None

\subsubsection{Access Routine Semantics}

\noindent new AdminT($id, pass$):
\begin{itemize}
\item transition: $adminId, password, layouts :=  id, pass, \langle \rangle$
\item output: $out := self$
\item exception: $exc := ((|id| = 0) \vee (|pass| = 0) \Rightarrow
AdminCreationException)$
\end{itemize}

\noindent getAdminId():
\begin{itemize} 
\item output: $out := adminId$
\item exception: none
\end{itemize}

\noindent getPassword():
\begin{itemize} 
\item output: $out := password$
\item exception: none
\end{itemize}

\noindent getLayout($layoutId$):
\begin{itemize} 
\item output: $out := \exists (i:\mathbb{N}|i < n : layout[i].getLayoutId() =
layoutId) \Rightarrow layout[i]$
\item exception: $exc := \neg \exists (i:\mathbb{N}|i < n :
layout[i].getLayoutId() = layoutId) \Rightarrow \\ LayoutNotFoundException$
\end{itemize}

\noindent changeLayout($layoutId, newSpot$):
\begin{itemize} 
\item transition: $layout := \langle i:\mathbb{N} | i < n :
layout[i].getLayoutSpotId() = spotId \Rightarrow \\
layout[i].changeElem(newSpot.getElemId(), newSpot) | true \Rightarrow layout[i]
\rangle$
\item exception: $exc := \neg \exists (i:\mathbb{N}|i < n :
layout[i].getLayoutId() = layoutId) \Rightarrow \\ LayoutNotFoundException$
\end{itemize}

\newpage

\section{MIS of Parking Lot Layout Module} 
\label{parkingLotLayout:Module}

\subsection{Template Module}
ParkingLotLayoutT

\subsection{Uses}
\hyperref[parkingLayoutElem:Module]{ParkingLayoutElemT}

\subsection{Syntax}

\subsubsection{Exported Constants}
None

\subsubsection{Exported Types}
ParkingLotLayoutT = ?

\subsubsection{Exported Access Programs}

\begin{tabular}{l l l l}
\hline
\textbf{Name} & \textbf{In} & \textbf{Out} & \textbf{Exceptions} \\
\hline
new ParkingLotLayoutT & String, $\mathbb{N}$ & ParkingLotLayoutT &
LayoutCreationException \\
setAllRoads & ~ & ~ & ~ \\
getLayoutId & ~ & String & ~ \\
getSize & ~ & $\mathbb{N}$ & ~ \\
changeElem & \begin{tabular}{@{}lc@{}}String, \\ ParkingLayoutElemT\end{tabular}
& ~ & ~ \\
getLayout & ~ & \begin{tabular}{@{}lc@{}}Seq of \\
ParkingLayoutElemT\end{tabular} & ~ \\
getElem & String & ParkingLayoutElemT & ElemNotFoundException \\
getElemIndex & String & $\mathbb{N}$ & ElemNotFoundException \\
\hline
\end{tabular}

\subsection{Semantics}

\subsubsection{State Variables}
$layoutId$: String \\
$n$: $\mathbb{N}$\\
$layout$: seq of ParkingLayoutElemT

\subsubsection{Environment Variables}
None

\subsubsection{State Invariant}
$|layout| = n^2$.

\subsubsection{Assumptions}
None

\subsubsection{Access Routine Semantics}

\noindent new ParkingLotLayoutT($id, size$):
\begin{itemize}
\item transition: $layoutId, n, layout :=  id, size, \langle \rangle$
\item output: $out := self$
\item exception: exc := $((|id| = 0) \Rightarrow AdminCreationException)$
\end{itemize}

\noindent setAllRoads():
\begin{itemize} 
\item output: $layout := i:\mathbb{N} | i < n^2 : layout || \langle new
ParkingLayoutElem("road_" + i, "road) \rangle $
\item exception: none
\end{itemize}

\noindent getLayoutId():
\begin{itemize} 
\item output: $out := layoutId$
\item exception: none
\end{itemize}

\noindent getSize():
\begin{itemize} 
\item output: $out := n$
\item exception: none
\end{itemize}

\noindent changeElem($elemId, newElem$):
\begin{itemize} 
\item transition: $layout := \langle i:\mathbb{N} | i < n :
layout[i].getElemId() = elemId \Rightarrow newElem | true \Rightarrow layout[i]
\rangle$
\item exception: none
\end{itemize}

\noindent getLayout():
\begin{itemize} 
\item output: $out := layout$
\item exception: none
\end{itemize}

\noindent getElem($elemId$):
\begin{itemize} 
\item output: $out := \exists (i:\mathbb{N}|i < n : layout[i].getLayoutId() =
elemId) \Rightarrow layout[i]$
\item exception: $exc := \neg \exists (i:\mathbb{N}|i < n :
layout[i].getLayoutId() = elemId) \Rightarrow \\ ElemNotFoundException$
\end{itemize}

\noindent getElemIndex($elemId$):
\begin{itemize} 
\item output: $out := \exists (i:\mathbb{N}|i < n : layout[i].getLayoutId() =
elemId) \Rightarrow i$
\item exception: $exc := \neg \exists (i:\mathbb{N}|i < n :
layout[i].getLayoutId() = elemId) \Rightarrow \\ ElemNotFoundException$
\end{itemize}

\newpage

\section{MIS of Parking Layout Element Module} 
\label{parkingLayoutElem:Module}

\subsection{Template Module}
ParkingElemT

\subsection{Uses}
None

\subsection{Syntax}

\subsubsection{Exported Constants}
None

\subsubsection{Exported Types}
ParkingElemT = ?

\subsubsection{Exported Access Programs}

\begin{tabular}{l l l l}
\hline
\textbf{Name} & \textbf{In} & \textbf{Out} & \textbf{Exceptions} \\
\hline
new ParkingElemT & String, String & ParkingElemT & ~ \\
getElemId & ~ & String & ~ \\
getType & ~ & String & ~ \\
\hline
\end{tabular}

\subsection{Semantics}

\subsubsection{State Variables}
$elemId$: String \\
$type$: String

\subsubsection{Environment Variables}
None

\subsubsection{State Invariant}
$type \in \{"spot", "road", "obstacle"\}$.

\subsubsection{Assumptions}
None

\subsubsection{Access Routine Semantics}

\noindent new ParkingElem($id, s, x, y$):
\begin{itemize}
\item transition: $elemId, type :=  id, s$
\item output: $out := self$
\item exception: exc := None
\end{itemize}

\noindent getElemId():
\begin{itemize} 
\item output: $out := elemId$
\item exception: none
\end{itemize}

\noindent getType():
\begin{itemize} 
\item output: $out := type$
\item exception: none
\end{itemize}

\newpage

\section{MIS of Parking Spot Module} 
\label{parkingSpot:Module}

\subsection{Template Module Inherits Parking Layout Element}
ParkingSpotT

\subsection{Uses}
\hyperref[parkingLayoutElem:Module]{ParkingLayoutElemT}

\subsection{Syntax}

\subsubsection{Exported Constants}
None

\subsubsection{Exported Types}
ParkingSpotT = ?

\subsubsection{Exported Access Programs}

\begin{tabular}{l l l l}
\hline
\textbf{Name} & \textbf{In} & \textbf{Out} & \textbf{Exceptions} \\
\hline
new ParkingSpotT & String, $\mathbb{R}$, $\mathbb{R}$ & ParkingSpotT & ~ \\
setEnabled & $\mathbb{B}$ & ~ & ~\\
setOccupied & $\mathbb{B}$ & ~ & ~\\
setReserved & $\mathbb{B}$ & ~ & ~\\
setHandicapped & $\mathbb{B}$ & ~ & ~\\
setAllProp & Seq of $\mathbb{B}$ & ~ & InvalidParamException\\
getEnabled & ~ & $\mathbb{B}$ & ~\\
getOccupied & ~ & $\mathbb{B}$ & ~\\
getReserved & ~ & $\mathbb{B}$ & ~\\
getHandicapped & ~ & $\mathbb{B}$ & ~\\
getAllProp & ~ & Seq of $\mathbb{B}$ & ~\\
\hline
\end{tabular}

\subsection{Semantics}

\subsubsection{State Variables}
$enabled$: $\mathbb{B}$\\
$occupied$: $\mathbb{B}$\\
$reserved$: $\mathbb{B}$\\
$handicapped$: $\mathbb{B}$\\

\subsubsection{Environment Variables}
None

\subsubsection{State Invariant}
None

\subsubsection{Assumptions}
None

\subsubsection{Access Routine Semantics}

\noindent new ParkingSpot($id$):
\begin{itemize}
\item transition: $spotId, type, enabled, reserved, handicapped, occupied := id,
"spot", \\
true, false, false, false$
\item output: $out := self$
\item exception: exc := None
\end{itemize}

\noindent setEnabled($e$):
\begin{itemize} 
\item transition: $enabled := e$
\item exception: none
\end{itemize}

\noindent setOccupied($o$):
\begin{itemize} 
\item transition: $occupied := o$
\item exception: none
\end{itemize}

\noindent setReserved($r$):
\begin{itemize} 
\item transition: $reserved := r$
\item exception: none
\end{itemize}

\noindent setHandicapped($h$):
\begin{itemize} 
\item transition: $handicapped := h$
\item exception: none
\end{itemize}

\noindent setAllProp($p$):
\begin{itemize} 
\item transition: $enabled, occupied, reserved, handicapped := p[0], p[1], p[2],
p[3]$
\item exception: $exc := |p| \neq 4 \Rightarrow InvalidParamException$
\end{itemize}

\noindent getEnabled():
\begin{itemize} 
\item output: $out := enabled$
\item exception: none
\end{itemize}

\noindent getOccupied():
\begin{itemize} 
\item output: $out := occupied$
\item exception: none
\end{itemize}

\noindent getReserved():
\begin{itemize} 
\item output: $out := reserved$
\item exception: none
\end{itemize}

\noindent getHandicapped():
\begin{itemize} 
\item output: $out := handicapped$
\item exception: none
\end{itemize}

\noindent getAllProp():
\begin{itemize} 
\item output: $out := \langle enabled, occupied, reserved, handicapped \rangle$
\item exception: none
\end{itemize}

\newpage

\section{MIS of Parking Stats Module} 
\label{parkingStats:Module}

\subsection{Module}
ParkingStatsT

\subsection{Uses}
\hyperref[parkingLotLayout:Module]{ParkingLotLayoutT},
\hyperref[parkingSpot:Module]{ParkingSpotT}

\subsection{Syntax}

\subsubsection{Exported Constants}
None

\subsubsection{Exported Types}
ParkingStatsT = ?

\subsubsection{Exported Access Programs}

\begin{tabular}{l l l l}
\hline
\textbf{Name} & \textbf{In} & \textbf{Out} & \textbf{Exceptions} \\
\hline
getStat & ParkingLotLayoutT, Seq of $\mathbb{B}$ & $\mathbb{N}$ &
InvalidParamException\\
\hline
\end{tabular}

\subsection{Semantics}

\subsubsection{State Variables}
None

\subsubsection{Environment Variables}
None

\subsubsection{State Invariant}
None

\subsubsection{Assumptions}
None

\subsubsection{Access Routine Semantics}

\noindent getStat($l, p$):
\begin{itemize} 
\item output: $out := +(\forall (e : ParkingSpotT | e \in l.getLayout() :
e.getType() = "spot" \land e.getAllProp() = p \Rightarrow 1 | true \Rightarrow
0))$
\item exception: $exc := |p| \neq 4 \Rightarrow InvalidParamException$
\end{itemize}

\newpage

\section{MIS of Navigation Module} 
\label{navigation:Module}

\subsection{Module}
NavigationT

\subsection{Uses}
\hyperref[parkingLotLayout:Module]{ParkingLotLayoutT}, 
\hyperref[parkingLayoutElem:Module]{ParkingLayoutElemT}

\subsection{Syntax}

\subsubsection{Exported Constants}
None

\subsubsection{Exported Types}
NavigationT = ?

\subsubsection{Exported Access Programs}

\begin{tabular}{l l l l}
\hline
\textbf{Name} & \textbf{In} & \textbf{Out} & \textbf{Exceptions} \\
\hline
findPath & ParkingLotLayoutT, String, String & Seq of String & NoPathException\\
\hline
\end{tabular}

\subsection{Semantics}

\subsubsection{State Variables}
None

\subsubsection{Environment Variables}
None

\subsubsection{State Invariant}
None

\subsubsection{Assumptions}
None

\subsubsection{Access Routine Semantics}

\noindent findPath($L, startId, stopId$):
\begin{itemize}
\item output: $out := \langle s : \text{String} \rangle$ that satisfies the
following:
\begin{itemize}
\item $out[0] = startId$.
\item $out[|out| - 1] = stopId$.
\item $out \subseteq \langle \forall (e : ParkingLayoutElemT | e \in
L.getLayout() : e.getElemId) \rangle$.
\item $\forall (i : \mathbb{N} | i < |out| - 2 : L.getElem(out[i]).getType() =
"Road")$.
\item $L.getElem(out[|out| - 1]).getType() = "Spot"$.
\item $\forall (i : \mathbb{N} | i < |out| - 1 : l.getElemIndex(out[i]) -
l.getElemIndex(out[i + 1]) \in \{1, -1, l.getSize(), -l.getSize()\})$.
\end{itemize}
\item exception: exc := No $out$ can satisfy the requirements $\Rightarrow
NoPathException$
\end{itemize}

\newpage

\section{MIS of Machine Learning Model Module} 
\label{machineLearningModel:Module}

\subsection{Module}
ModelT

\subsection{Uses}
\hyperref[cameraCapture:Module]{CameraCaptureT}

\subsection{Syntax}

\subsubsection{Exported Constants}
None

\subsubsection{Exported Types}
ModelT = ?

\subsubsection{Exported Access Programs}

\begin{tabular}{l l l l}
\hline
\textbf{Name} & \textbf{In} & \textbf{Out} & \textbf{Exceptions} \\
\hline
setModel & String & ~ & ~\\
setInput & String, String & ~ & ~ \\
getResult & ~ & Seq of $\mathbb{B}$ & ~ \\
\hline
\end{tabular}

\subsection{Semantics}

\subsubsection{State Variables}
None

\subsubsection{Environment Variables}
model: the pre-trained machine learning model in the machine. The model will be
deserialized as a python object.

\subsubsection{State Invariant}
None

\subsubsection{Assumptions}
None

\subsubsection{Access Routine Semantics}

\noindent setModel($pickleAdd$):
\begin{itemize}
\item transition: $modelAddress, model := pickleAdd, load(pickleAdd)$
\item exception:  $exc := \neg \exists (address : String | \neg address.exist())
\Rightarrow \\ InvalidAddressException$
\end{itemize}

\noindent setInput($inputURL, inputTypes$):
\begin{itemize}
\item transition: $ InputURL, Inputtype := inputURL,inputTypes$
\item exception: none
\end{itemize}

\noindent getResult():
\begin{itemize}
    \item output: $out := model.predict()$
    \item exception: none
\end{itemize}

\newpage

\section{MIS of Camera Capture Module}
\label{cameraCapture:Module}

\subsection{Module}

CameraCaptureT

\subsection{Uses}
None

\subsection{Syntax}

\subsubsection{Exported Constants}
None

\subsubsection{Exported Types}
CameraCaptureT = ?

\subsubsection{Exported Access Programs}

\begin{tabular}{l l l l}
\hline
\textbf{Name} & \textbf{In} & \textbf{Out} & \textbf{Exceptions} \\
\hline
getLatestFrame & String & Image File & \\
getLatestClip & String & Video File & \\
\hline
\end{tabular}

\subsection{Semantics}

\subsubsection{State Variables}
\textit{currentFrame}: Image File \\
\textit{currentClip}: Video File

\subsubsection{Environment Variables}
picamera

\subsubsection{Assumptions}
None

\subsubsection{Access Routine Semantics}

\noindent getLatestFrame(\textit{directory}):
\begin{itemize}
\item transition: \textit{currentFrame} := \textit{picamera.capture(directory)}
\item output: \textit{out} := \textit{currentFrame} 
\item exception: None
\end{itemize}

\noindent getLatestClip():
\begin{itemize}
\item transition: \textit{currentClip} := \textit{picamera.record(directory)}
\item output: \textit{out} := \textit{currentClip} 
\item exception: None
\end{itemize}

\newpage

\section{MIS of User Action Handler module} 
\label{userActionHandler:Module}

\subsection{Module}
UserActionHandlerT

\subsection{Uses}

\hyperref[navigation:Module]{NavigationT}, \hyperref[auth:Module]{AuthT},
\hyperref[admin:Module]{AdminT}, \hyperref[machineLearningModel:Module]{ModelT},
\hyperref[parkingStats:Module]{ParkingStatsT},
\hyperref[view:Module]{ViewT} 

\subsection{Syntax}

\subsubsection{Exported Constants}
none

\subsubsection{Exported Types}
UserHandlerT = ?

\subsubsection{Exported Access Programs}

\begin{tabular}{l l l l}
\hline
\textbf{Name} & \textbf{In} & \textbf{Out} & \textbf{Exceptions} \\
\hline
handleChangeLayout & String, ParkingLayoutElemT & ~ & ~ \\
handleParkingStats& ParkingLotLayoutT, Seq of $\mathbb{B}$ & $\mathbb{N}$ & ~ \\
handleCheckAvailability & ~ & Seq of $\mathbb{B}$ & ~\\
handleFindPath & ParkingLotLayoutT, String, String & Seq of String & \\
handleAuth & Sting, String & ~ $\mathbb{B}$ & ~ \\
\hline
\end{tabular}

\subsection{Semantics}

\subsubsection{State Variables}
None

\subsubsection{Environment Variables}
None

\subsubsection{State Invariant}
None

\subsubsection{Assumptions}
None

\subsubsection{Access Routine Semantics}

\noindent handleCheckAvailability():
\begin{itemize}
    \item output: $out := ModelT.getResult()$
    \item exception: none
\end{itemize}

\noindent handleChangeLayout($st, parkingele$):
\begin{itemize}
    \item output: $out := AdminT.changeLayout(st, parkingele)$
    \item exception: none
\end{itemize}


\noindent handleParkingStats($layout, boolArray$):
\begin{itemize}
    \item output: $out := ParkingStatsT.getStat(st, parkingele)$
    \item exception: none
\end{itemize}



\noindent handleFindPath($l, startID, stopId$):
\begin{itemize}
    \item output: $out := navigationT.findPath(layout, String, String)$
    \item exception: none
\end{itemize}

\noindent handleAuth($id, pass$):
\begin{itemize}
    \item output: $out := AuthT.authenticateUser(id,pass)$
    \item exception: none
\end{itemize}

\newpage

\section{MIS of View module}
\label{view:Module}

\subsection{Module}
ViewT

\subsection{Uses}
\hyperref[parkingLotLayout:Module]{ParkingLotLayoutT}, 
\hyperref[parkingLayoutElem:Module]{ParkingLayoutElemT} 

\subsection{Syntax}

\subsubsection{Exported Constants}
None

\subsubsection{Exported Types}
viewT = ?

\subsubsection{Exported Access Programs}

\begin{tabular}{l l l l}
\hline
\textbf{Name} & \textbf{In} & \textbf{Out} & \textbf{Exceptions} \\
\hline
initLogin & ~ & ~ & ~ \\
initPage & ~ & ~ & ~ \\

\hline
\end{tabular}

\subsection{Semantics}

\subsubsection{State Variables}
None

\subsubsection{Environment Variables}
$win$: two dimensional sequence of coloured pixels

\subsubsection{State Invariant}
None

\subsubsection{Assumptions}
None

\subsubsection{Access Routine Semantics}

\noindent initLogin():
\begin{itemize} 
\item transition: modify win with the following:
\begin{itemize} 
\item a text input field for a user name.
\item a text input field for a password.
\item a button to confirm the text inputs.
\end{itemize}
\item exception: none
\end{itemize}

\noindent initPage():
\begin{itemize} 
\item transition: modify win with the following:
\begin{itemize} 
\item use $showLayout()$ based on the location.
\item use $showLayout()$ to show a path.
\item a button to confirm logout.
\item a panel to show parking stats.
\end{itemize}
\item exception: none
\end{itemize}

\subsubsection{Local Functions}

\noindent showLayout($L$):
\begin{itemize} 
\item transition: modify win so the elements $e :ParkingLotLayoutT$ of $L$ are
displayed on a $L.getSize()$ by $L.getSize()$ grid with the following table: 

\begin{tabular}{l l l l l l}
\hline
$getType()$ & $getEnabled()$ & $getHandicapped()$ & $getReserved()$ &
$getOccupied()$ & character \\
\hline
road & ~  & ~  & ~  & ~  & Black +\\
obstacle & ~  & ~  & ~  & ~  & Black O\\
spot & true  & false  & false  & false  & Green S\\
spot & true  & false  & false  & true  & Red S\\
spot & true  & false  & true  & false  & Green R\\
spot & true  & false  & true  & true  & Red R\\
spot & true  & true  & false  & false  & Green H\\
spot & true  & true  & false  & true  & Red H\\
spot & false  & ~  & ~  & ~  & Grey S\\
\hline
\end{tabular}
\end{itemize}

\newpage

\noindent showPath($L, startId, stopId$):
\begin{itemize} 
\item transition: modify win first with $showLayout(L)$. \\
$\forall (s : String | s \in NavigationT.findPath(L, startId, stopId) : \\
\text{bolden the character on } win \text{ corresponding to } L.getElem(s))$ 
\item exception: none
\end{itemize}

\newpage

% End of the sample
\bibliographystyle {plainnat}
\bibliography{../../../refs/References}

\newpage

\section{Appendix} \label{Appendix} N/A

\end{document}