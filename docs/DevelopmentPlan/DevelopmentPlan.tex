\documentclass[12pt,letterpaper]{article}
\usepackage[utf8]{inputenc}
\usepackage[margin=1in]{geometry}
\usepackage[titletoc,title]{appendix}
\usepackage{graphicx}
\usepackage{booktabs}
\usepackage{hyperref}
\usepackage{tabularx}
\usepackage{indentfirst}
\usepackage{xcolor}
\usepackage[normalem]{ulem}

\title{Development Plan\\\progname}

\author{\authname}

\date{\today}

%% Comments

\usepackage{color}

\newif\ifcomments\commentsfalse %displays comments
%\newif\ifcomments\commentsfalse %so that comments do not display

\ifcomments
\newcommand{\authornote}[3]{\textcolor{#1}{[#3 ---#2]}}
\newcommand{\todo}[1]{\textcolor{red}{[TODO: #1]}}
\else
\newcommand{\authornote}[3]{}
\newcommand{\todo}[1]{}
\fi

\newcommand{\wss}[1]{\authornote{blue}{SS}{#1}} 
\newcommand{\plt}[1]{\authornote{magenta}{TPLT}{#1}} %For explanation of the template
\newcommand{\an}[1]{\authornote{cyan}{Author}{#1}}

%% Common Parts

\newcommand{\progname}{ProgName} % PUT YOUR PROGRAM NAME HERE
\newcommand{\authname}{Team \#, Team Name
\\ Student 1 name
\\ Student 2 name
\\ Student 3 name
\\ Student 4 name} % AUTHOR NAMES                  

\usepackage{hyperref}
    \hypersetup{colorlinks=true, linkcolor=blue, citecolor=blue, filecolor=blue,
                urlcolor=blue, unicode=false}
    \urlstyle{same}
                                


\begin{document}

\maketitle

\newpage
\begin{table}[hp]
\caption{Revision History} \label{TblRevisionHistory}
\begin{tabularx}{\textwidth}{lll}
\toprule
\textbf{Date} & \textbf{Developer(s)} & \textbf{Change}\\
\midrule
Sep 22, 2022 & Albert, Almen, David, Gary, Jonathan, Kabishan & Revision 0\\
\bottomrule
\end{tabularx}
\end{table}

\newpage
\tableofcontents
\newpage
\listoftables
\newpage

This document is the Development Plan for the project, Park'd, an application
that aims to help the average driver quickly find an open parking space,
especially on busy days with few spots available.
%%%%%%%%%%%%%%%%%%%%%%%%%%%%%% Almen %%%%%%%%%%%%%%%%%%%%%%%%%%%%%%
\section{Team Meeting Plan}
\label{TeamMeetingPlan}
The following is the plan for when, the location, frequency, and duration of the
team meetings:
\begin{table}[hp]
\begin{tabularx}{\textwidth}{|l|l|X|}
\toprule
\textbf{Where} & \textbf{When} & \textbf{Duration}\\
\midrule
H.G. Thode Library & Every Wednesday at 10:30AM & 1 Hour\\
Discord & Every Monday at 9:00PM & 2 Hour\\
\bottomrule
\end{tabularx}
\caption{Team Meeting Plan} \label{tab:teamMeetingPlan}
\end{table}

\subsection{Meeting Roles}
\label{meetingRoles}
The table below, \nameref{meetingRolesTable}, identifies the assignment of roles
during the meeting.
\begin{table}[hp] 
    \centering
    \begin{tabularx}{\textwidth}{|X|X|}
        \toprule
        \textbf{Meeting Role} & \textbf{Name}\\
        \midrule
        Participant & Everyone\\
        Chair & Jonathan Yapeter\\
        Scribe & Albert Zhou\\
        Timekeeper & Albert Zhou\\
        \bottomrule
    \end{tabularx}
\caption{Meeting Roles} 
\label{meetingRolesTable}
\end{table}

The responsibilities of the meeting roles identified in
\nameref{meetingRolesTable} are as follows.
\subsubsection{Participant}
\begin{enumerate}
    \item Understand the agenda and the purpose of the meeting 
    \item Propose topics prior to the meeting that needs to be discussed
    \item Contribute to the discussions of all agenda items
\end{enumerate}
\subsubsection{Chair}
\begin{enumerate}
    \item Guides the group through the predetermined agenda
    \item Facilitates respectful discussion among all participants
    \item Establish equal opportunities to speak
    \item Clarify conclusions and next steps at the end of each meeting
\end{enumerate}
\subsubsection{Scribe}
\label{scribe}
\begin{enumerate}
    \item Create the meeting minutes prior to the meeting
    \item Notify all participants of the agenda before the meeting
    \item Note all the decisions, conclusions, and action items presented during
    the meeting
    \item Compile notes in a standard template with consistent formatting
\end{enumerate}
\subsubsection{Timekeeper}
\begin{enumerate}
    \item Monitors, records, and manages the time it takes to accomplish an item
    in the agenda
\end{enumerate}

\subsection{Rules of Agenda}
The following are the rules of agenda for all in-person and online team
meetings:
\begin{enumerate}
    \item An agenda must be created prior to every meeting. The following
    guidelines should be followed when creating the agenda:
    \begin{itemize}
        \item First topic should be to review the agenda
        \item An estimated allotted time for each topic
        \item Topics must be formulated as questions
        \item Identify the people responsible for each topic
        \item Seek input from all team members
    \end{itemize}
    \item All members must be present during the meeting and attendance will be
    recorded
    \begin{itemize}
        \item Scheduled absences must be communicated to the rest of the team at
        least one day in advance of the meeting
    \end{itemize}
    \item Meeting minutes will be taken during every meeting by Albert, our
    \nameref{scribe}.
    \item Prior to current agenda, any conflicts and/or discussions from the
    previous action items must be addressed 
    \item All members must work and communicate respectfully to each other
    during the meeting
\end{enumerate}

%%%%%%%%%%%%%%%%%%%%%%%%%%%%%% Almen %%%%%%%%%%%%%%%%%%%%%%%%%%%%%%
\section{Team Communication Plan}
The meetings mentioned in \nameref{TeamMeetingPlan} will be conducted in-person
at H.G. Thode Library on campus or on Discord. Members are to be seated together
in the first floor of Thode or in the appropriate voice channel by the start of
the meeting. 

To facilitate communication outside of designated meeting times, a Messenger
group chat with all members of caPstOneGroup has been created. The expectation
for all team members is to check the Messenger group chat at a minimum of twice
per day and will respond to mentions at their earliest convenience. Any urgent
issues that arise regarding milestones shall be brought up immediately in the
Messenger group chat. 

Git issues will be created with the appropriate Priority/Severity label and
assigned to the team members involved if there are any tasks needed to be done
or bugs needed to be fixed. Discussions regarding the project code will be
predominantly done through these Git issues, however, can be reinforced in the
Messenger group chat.

\begin{table}[hp]
\begin{tabularx}{\textwidth}{|l|X|X|X|}
\toprule
\textbf{Name} & \textbf{Discord} & \textbf{macid} & \textbf{GitHub}\\
\midrule
Albert Zhou & Ferman & zhouj103 & albertzevanescent\\
Almen Ng & Mumbojumbo & nga18 & almen-ng\\
David Yao & d i n g u s & yaod9 & davidwyao\\
Gary Gong & Shawnty & gongc12 & GaryGong27\\
Jonathan Yapeter & yapeter & yapetej & JAYapeter\\
Kabishan Suvendran & midnightmarauder & suvendrk & Kabishan\\
\bottomrule
\end{tabularx}
\caption{Team Meeting Plan} \label{tab:teamMeetingPlan}
\end{table}

%%%%%%%%%%%%%%%%%%%%%%%%%%%%%% Almen %%%%%%%%%%%%%%%%%%%%%%%%%%%%%%
\section{Team Member Roles}
The table below, \nameref{tab:memberRoles}, identifies the assignment of main
roles of the team.
\begin{table}[hp!] 
    \centering
    \begin{tabularx}{\textwidth}{|X|X|}
        \toprule
        \textbf{Name} & \textbf{Role(s)}\\
        \midrule
        All                         & Developer\\
                                    & QA Tester\\
                                    & DevOps \\
        \hline
        Albert Zhou                 & Scribe\\
        \hline
        Almen Ng                    & Project Lead\\
        & Team Liaison\\
        \hline
        David Yao                   & Documentation/LaTeX Expert\\
        \hline
        Gary Gong                   & Machine Learning Expert \\
        \hline
        Jonathan Yapeter            & Git Expert \& Maintainer\\
        \hline
        Kabishan Suvendran          & User Interface Expert\\
        \bottomrule
    \end{tabularx}
\caption{Member Roles} \label{tab:memberRoles}
\end{table}

\subsection{Team Member Role Descriptions}
This subsection will provide descriptions on all the team member roles
identified and assigned in \nameref{tab:memberRoles}.
\subsubsection{Developer}
\begin{itemize}
    \item Design and builds application features
    \item Produce clean and efficient code based on specifications 
    \item Troubleshoot and resolve any bugs that may arise 
    \item Propose and execute improvements to the code base
    \item Provide technical documentation to all code for reference
\end{itemize}

\subsubsection{QA Tester}
\begin{itemize}
    \item Test new and existing features
    \item Automate unit and integration tests
    \item Report any bugs or failures and suggest fixes 
\end{itemize}

\subsubsection{DevOps}
\begin{itemize}
    \item Automate repetitive tasks through GitHub actions/pipelines to decrease
    the workload of team members
    \item Facilitate performance testing and benchmarking to evaluate how
    reliable the system runs
    \item Introduce processes, tools and methodologies to support the software
    development lifecycle
\end{itemize}

\subsubsection{Scribe}
\begin{itemize}
    \item Refer to \nameref{scribe} in \nameref{meetingRoles} for the
    responsibilities of a Scribe
\end{itemize}

\subsubsection{Project Lead}
\begin{itemize}
    \item Ensure the team is on track with completing all milestones outlined in
    the
    \href{https://gitlab.cas.mcmaster.ca/courses/capstone/-/blob/main/CourseOutline/Capstone_Outline.pdf}{capstone
    course outline} by the deadline
    \item Devise plans that support project goals
    \item Address any bottlenecks or conflicts that may arise 
\end{itemize}

\subsubsection{Team Liaison}
\begin{itemize}
    \item Disseminate information given by the instructors to the rest of the
    team
\end{itemize}

\subsubsection{Documentation/LaTeX Expert}
\begin{itemize}
    \item Formats LaTeX documents for milestones with written reports
    \item Fix any compilation errors regarding LaTeX documents that may arise
    \item Create new LaTeX commands that may be needed
    \item Create and enforce a standard for documenting code and features
\end{itemize}

\subsubsection{Machine Learning Expert}
\label{mlexpert}
\begin{itemize}
    \item Answer any questions related to Machine Learning, like Computer Vision
    \item Lead the building, training and deploying of models necessary for the
    application
\end{itemize}

\subsubsection{Git Expert \& Maintainer}
\begin{itemize}
    \item Answer any questions related to Git, like PRs and Issues
    \item Track all ongoing Git issues/PRs/tasks and make sure that they are
    reviewed and merged in a timely manner
    \item Ensure that all changes made in the Git repository are following
    correct naming conventions, coding guidelines, and follow the proper
    workflow as mentioned in \nameref{workflowPlan}.
\end{itemize}

\subsubsection{User Interface Expert}
\begin{itemize}
    \item Lead the design of the application to support user interface and user
    experience principles
\end{itemize}


%%%%%%%%%%%%%%%%%%%%%%%%%%%%%% Kabishan %%%%%%%%%%%%%%%%%%%%%%%%%%%%%%
\section{Workflow Plan}
\label{workflowPlan}
We will use GitHub to maintain the code, the documentation for said code and the
milestone deliverables, including this report. All of the code or documentation
that will be viewed and marked by the teaching assistant will be merged into the
main branch before each milestone's due date. Changes will not be committed
directly onto the main branch, since the main branch will be protected from
erroneous pushes. Instead, each developer will create their own branch, create a
PR and assign another team member to review and approve their changes before
they are merged into the main branch. For changes related to the code, in
addition to team approvals, the code must also pass all of the unit tests that
were set up as a part of the Git Workflow, and Pylint will be used for source
code analysis to detect any errors and enforcing coding standards. For changes
related to LaTeX files, the files must compile and produce a PDF document before
the changes are merged. 

To manage issues, including template issues, issue classifications, and general
inquiries about certain aspects of the code, our team will employ a project
management tool, such as Zenhub, on which, we will use a project board to create
and manage issues. On this board, we will sort each and every issue into several
categories:
\begin{itemize}
    \item Untriaged (not assigned a priority or time frame for completion) \item
    To Do (work that is yet to be worked on)
    \item In Progress (being actively developed)
    \item Code Review (code completed and PR opened)
    \item Done (code approved and merged into main)
\end{itemize}

We will also use labels and tags, such as front-end, back-end, documentation and
machine learning, to indicate which issues correspond to which component of the
project. Using GitHub Issue tracker, we will reference specific issues, tag
developers to notify them of the issue, set up meetings to discuss the issue,
and finally link the PR that resolves the issue.

As a part of our Workflow Plan, we will maintain a Contributor's Guide, in
which, we will mention the specifics of pushing quality code and documentation.
For example, each function, method and class should be documented based on their
parameters, purpose, and return value. The exact formatting of the documentation
will be based on the style guide for the chosen programming language.
Furthermore, the documentation must be succinct, coherent, and must not contain
spelling or grammar mistakes. This is also required of LaTeX documents, as well
as project deliverables. Furthermore, another aspect of the Contributor's Guide
is how pull requests are opened. Each pull request must be opened with a
description that mentions what the code changes will do and it should be
assigned to at least one other team member for approval.

%%%%%%%%%%%%%%%%%%%%%%%%%%%%%% David %%%%%%%%%%%%%%%%%%%%%%%%%%%%%%
\section{Proof of Concept Demonstration Plan}

The Proof of Concept (POC) demonstration plan highlights the risks we have
discovered in the creation of our project, as well as how the demonstration
should show that those risks can be overcome.

\subsection{Risks}
\begin{itemize}
    \item \textbf{Finding a suitable parking area to test}: A parking lot
    administrator will need to agree to either make their video equipment
    available for use in the program, or allow such equipment to be installed. A
    video feed is integral to the proposed functioning of our program, because
    of the need for machine vision in analyzing parking activity.
    \item \textbf{Computation time}: Machine learning is computationally
    expensive, so a poorly optimized program may not respond with the promptness
    required for a real-time driving aid platform. 
\end{itemize}

\subsection{POC Demo}
Our proof of concept will demonstrate the program's capability to detect the
state of a parking space. It will be a recorded run with a single parking spot.
The program will first poll the state of the spot while empty, and then a
vehicle will pull into the spot once the correct status has been determined. The
demonstration's relation to the given risks is as follows:
\begin{itemize}
    \item \textbf{Finding a suitable parking area to test}: Our need for a
    suitable parking spot will be fulfilled by this time if the POC
    demonstration is to go ahead as planned. If no such place is found, some
    group members will volunteer their driveways as simulated parking spots. In
    addition, given our program's usage of video feeds, we are also prepared to
    proceed using publicly available online feeds of parking lots.
    \item \textbf{Computation time}: Only the baseline functionality for our
    program, the detection of parking spot occupation, is being tested as part
    of the POC demonstration. By showing a baseline for the computation time
    that we should expect from this task, we will then decide what additional
    features we can afford to add from our stretch goals. If even the baseline
    task is too intensive, we will explore either moving the back-end to a more
    powerful machine, or finding a non-machine learning algorithm to accomplish
    the task.
\end{itemize}


%%%%%%%%%%%%%%%%%%%%%%%%%%%%%% Gary %%%%%%%%%%%%%%%%%%%%%%%%%%%%%%
\section{Technology}
The following are the technologies we will be using for our project.
\begin{itemize}
    \item Programming languages
    \begin{itemize}
        \item We will be using Python for training the machine learning model as
        well as making a back-end service 
        \item JavaScript and HTML CSS will be used for developing the front-end
        web page of our project
    \end{itemize}
    \item IDE
    \begin{itemize}
        \item Front-end: Visual studio code
        \item Back-end/ML: PyCharm
    \end{itemize}
    \item Linter and Code coverage measure tools
    \begin{itemize}
        \item We will be using Pylint and flake 8 for different purposes
        including check for errors, tries to enforce a coding standard, looks
        for code smells
    \end{itemize}
    \item Testing
    \begin{itemize}
        \item Pytest will be used for making unit test cases
    \end{itemize}
    \item Libraries
    \begin{itemize}
        \item OpenCV,TensorFlow Numpy will be used for training neural networks
        and configuring training data sets.
        \item Flask will be used for creating back-end services of our project
    \end{itemize}
    \item Performance measuring tools
    \begin{itemize}
        \item Not applicable
    \end{itemize}
    \item DevOps
    \begin{itemize}
        \item Basic DevOps plans including static code analysis, linting,
        running unit test cases using Pylint and Pytest will be conducted when
        pull requests are created.
    \end{itemize}
    \item Documentation
    \begin{itemize}
        \item Doxygen will be used for generating documentations for the
        back-end services and APIs
        \item Development document will be edited through overleaf LaTex editor
        in a collaborative fashion
    \end{itemize}
    \item Hardware
    \begin{itemize}
        \item Cameras from the parking lot 
    \end{itemize}
\end{itemize}

%%%%%%%%%%%%%%%%%%%%%%%%%%%%%% Gary %%%%%%%%%%%%%%%%%%%%%%%%%%%%%%
\section{Coding Standard}
The back-end part of the project and the algorithm part for training the machine
learning model will follow the \href{https://peps.python.org/pep-0008/} {PEP8}
python coding standard. The linter in the workflow plan will ensure the
standards are followed for the code written.

\section{Project Scheduling}
Project scheduling is based on the course deadlines and will be tracked using
\href{https://www.zenhub.com/}{Zenhub}, a project management tool that is
directly integrated into GitHub through an extension. Zenhub's roadmap tool
displays a timeline and blocks during which given tasks should be completed.
Roadmaps will also integrate data from our issues. Zenhub also has a counter
feature that allows you to approximate an estimated amount of time it takes to
complete an issue which would help us get a better idea of what our timeline
would look like.

To determine when a major milestones are due, we will take the course deadlines,
discuss approximately how much time it will take to complete each task, and
define a definitive date to complete the major milestones by while allocating
some time prior to the actual course deadlines for any spillover.

For decomposing of larger tasks into smaller ones, we will discuss what is
needed for those tasks, write out issues for each individual task, then create a
\href{https://blog.zenhub.com/working-with-epics-in-github/}{Zenhub epic}, which
is a way to group all related sub tasks represented by issues into one
overarching issue.

In terms of assigning tasks, we will have a discussion on who is best suited to
accomplish the task while also providing opportunities for other members to
contribute based on their interest. Another factor that need to be considered is
the amount of time each person has per week to accomplish a task. Using the
Zenhub counter feature, we will be able to divide the tasks up so each member
will spend around 9-12 hours a week for the course. For example, if a task
involves computer vision and it would take 9 hours to complete, the
\nameref{mlexpert} will only be assigned this task and potentially one smaller
task that takes less than 3 hours.

\end{document}